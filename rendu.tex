\documentclass{report}

\usepackage[french]{babel}
\usepackage[utf8]{inputenc}
\usepackage[T1]{fontenc}
\usepackage{amsmath}
\usepackage{listings}

\title{Mini-projet : Alignement de séquences LU3IN003 - Sorbonne Université}
\author{GAMA Gustavo \\ EL BEBLAWY Rami}
\date{}

\begin{document}
\maketitle


\subsection*{Question 1}
Si $(\bar{x},\bar{y})$ et $(\bar{u},\bar{v})$ sont respectivement des alignements de $(x,y)$ et $(u,v)$ alors $(\bar{x}.\bar{u},\bar{y}.\bar{v})$ est un alignement de $(x.u, y.v)$ car:

- d'après (i), $\pi(\bar{y})=y$ et $\pi(\bar{v})=v$ donc $\pi(\bar{y}\cdot \overline{v}) = y\cdot v$.

- et d'après (ii), $\pi(\bar{y})= y$ et $\pi(\bar{v}) = v$ donc $\pi(\bar{y}\cdot \bar{v}) = y\cdot v$.

- si $(\bar{x},\bar{y})$ et $(\bar{u},\bar{v})$ sont des alignements, alors $|\bar{x}|=|\bar{y}|$ et $|\bar{u}|=|\bar{v}|$, donc $(\bar{x}.\bar{u}) = |\bar{x}|+|\bar{u}|$ et $(\bar{y}.\bar{v}) = |\bar{y}|+|\bar{v}|$

- il faut que (iv) soit respectée c'est-à-dire(définition): * $\forall i \in [1..|\bar{x}|]$, $\bar{x_{i}}\ne$ --- ou $\bar{y_{i}}\ne$ --- et $\forall i \in [1..|\bar{u}|], \bar{u_{i}}\ne -$ ou $\bar{v_{i}}\ne$ ---

Donc $\forall i\in [1..(|\bar{x}|+|\bar{u}|)]$, $((\overline{x\cdot u})_{i} \ne$ --- ou $(\overline{y\cdot v})_{i}$

\subsection*{Question 2}
La longueur maximale d'un aligement de $(x,y)$ est $n+m$ car on a alors un gap en face de chaque lettre, pour chacun des mots. On ne peut avoir deux gaps face à face, on ne peut donc augmenter la longueur du mot. Exemple avec les mots A et AG de taille n = 1 et m = 2
\begin{center}
$A--$

$-AG$
\end{center}
On a bien une longueur de $3 = n + m $.

\subsection*{Question 3}
En ajoutant k gaps à x, on obtient un mot de taille $n+k$. On va placer k gaps parmis ces $n+k$ places. On va donc choisir k parmis $n+k$, le nombre de mots $\bar{x}$ obtenus est donc $\binom{n+k}{k}$.

\subsection*{Question 4}
Une fois ajoutés k gaps à x pour obtenir $\bar{x}$, on a un mot de taille $n+k$. Afin de l'aligner avec y, il faut que $\bar{y}$ soit aussi de taille $n+k$. Or, y est de taille m. Il faut donc ajouter $n+k-m$ gaps à y.

Le nombre de mots possible pour $\bar{x}$ est $\binom{n+k}{k}$, et le nombre de mots possible pour $\bar{y}$ est  $\binom{n+k}{n+k-m}$. Or, on doit enlever les mots où des gaps se superposent. Il y a donc k choix en moins pour placer les gaps de $\bar{y}$. On obtient donc, pour un k donné, $\binom{n+k}{k}*\binom{n}{n+k-m}$ combinaisons possibles.

On va ajouter à $n$ de 0 à, au plus, $m$ gaps (on obtient ainsi le mot de taille maximale $n+m$).

Le nombre d'alignements possible pour $(x,y)$ est donc $ \sum_{k=0}^{k=m}\binom{n+k}{k}\binom{n}{n+k-m}$

En utilisant un calculateur en ligne, on trouve 298 199 265 alignements possibles pour $|x|=15$ et $|y|=10$.

\subsection*{Question 5}
On peut trouver l'alignement de coût minimal en même temps que l'énumération de tous les alignements possibles à l'aide d'une variable qui stocke le minimum obtenu à chaque fois, et l'aligement qui en est à l'origine?

Afin de calculer la distance d'édition de deux mots, il faut calculer la distance entre chaque lettre. Il y a $n+k$ lettres, c'est donc une opération en $O(n+k)=O(n)$(car $n>m\ge k$). On va effectuer cette opération pour chaque combinaison possible, la complexité obtenu est donc $O(n*\sum_{k=0}^{k=m}\binom{n+k}{k}\binom{n}{n+k-m})$, soit $O(n!)$.


\subsection*{Question 6}
Il n'est nécéssaire de garder en mémoire uniquement les 2 mots en cours d'analyse et le meilleur alignement trouvé jusqu'ici. On a donc une complexité spatiale en $O(n)$.


\clearpage

\subsection*{Question 7}
Si $\bar{u}_{l} = $ ---, alors $\bar{v}_{l} = y_{j}$ , si  $\bar{v}_{l} = $ ---, alors $\bar{u}_{l} = x_{i}$ . Si $\bar{u}_{l} \ne $ --- et $\bar{v}_{l} \ne $ ---, alors $\bar{u}_{l} = x_{i}$ et $\bar{v}_{l} = y_{j}$, car on ne peut changer l'ordre des lettres et on ne peut aligner deux gaps. Ainsi, la dernière lettre de $Al(i,j)$ est forcément, soit ---, soit $x_{i}$ (respectivement $y_{j}$).

\subsection*{Question 8}
Si $\bar{u}_{l} = $ --- ou $\bar{v}_{l} = $ ---, alors $C(\bar{u}, \bar{v}) = C(\bar{u}_{[1..l-1]}, \bar{v}_{[1..l-1]}) + c_{ins/del}$.

Sinon, $C(\bar{u}, \bar{v}) = C(\bar{u}_{[1..l-1]}, \bar{v}_{[1..l-1]}) + c_{sub}(x_{i},y_{j})$.


\subsection*{Question 9}
$$
D(i,j)= min\left\{
    \begin{array}{ll}
        D(i-1,j-1) + c_{sub}(x_{i},y_{j}) \\
        D(i-1,j) +  c_{del}\\
        D(i,j-1) +  c_{ins} 
    \end{array}
\right.
$$

\subsection*{Question 10}
$D(0,0)=0$  car il n'y a pas de ``coût initial'' pour commencer l'alignement.

\subsection*{Question 11}
Pour $j\in[1..m]$, on a $D(0,j)=c_{ins}*j$ car cela veut dire qu'il n'y aucune lettre de $ \Sigma$ dans $\bar{u}$, donc uniquement des insertions.

De même, pour  $i\in[1..n]$, on a $D(i,0)=c_{del}*i$ car cela veut dire qu'il n'y a  aucune lettre de $ \Sigma$ dans $\bar{v}$, donc uniquement des suppressions.

\clearpage

\subsection*{Question 12}
\begin{lstlisting}[
  mathescape,
]
DIST_1(x,y):

    Pour i allant de 0 a n:

        Pour j allant de 0 a m :
        
            Si i = 0:
                Si j = 0:
                    D[0,0] $\gets$ 0
                Sinon:
                    D[0,j] $\gets$ $j*c_{ins}$
        
            Sinon:
                Si j = 0:
                    D[i,0] $\gets$ $i*c_{del}$
                    
                Sinon :
                    c $\gets D[i-1,j-1] + c_{sub}(x_{i},y_{j})$
                    
                    Si D[i-1,j] + $c_{del}$ < c:
                        c $\gets D[i-1,j] + c_{del}$

                    Si D[i,j-1] + $c_{ins}$ < c:
                        c $\gets D[i,j-1] + c_{ins}$
                        
                    D[i,j] $\gets$ c
                    
        fin pour
    fin pour
    
    return D
\end{lstlisting}

\clearpage

\subsection*{Question 13}
L'algorithme DIST\_1 prend en mémoire un tableau imbriqué de taille n*m, sa complexité spatiale est donc de l'ordre $O(nm)$.

\subsection*{Question 14}
On a deux boucles imbriquées allant jusqu'à n/m. Sa complexité temporelle est donc de l'ordre de $O(max(n,m)^{2})$.

\subsection*{Question 15}
Si on a $j>0$ et $D(i,j) = D(i,j-1) + c_{ins}$, cela signifie que le meilleur coût de l'alignement de $(x_{[1..u]},y_{[1..j]})$ est obtenu à partir de l'aligement  $(x_{[1..u]},y_{[1..j-1]})$ en ajoutant une insertion. Cela se traduit, donc, d'après l'énoncé, à rajouter un gap dans $x$, et se traduit mathématiquement par 
$$ \forall(\bar{s},\bar{t}) \in Al*(i,j-1),  (\bar{s}\cdot-,\bar{t}\cdot y_{j}) \in Al*(i,j) $$
On applique les même raisonnement pour les deux autres expressions, et on trouve ainsi
$$ \forall(\bar{s},\bar{t}) \in Al*(i-1,j),  (\bar{s}\cdot x_{i},\bar{t}\cdot-) \in Al*(i,j) $$
$$ \forall(\bar{s},\bar{t}) \in Al*(i-1,j-1), (\bar{s}\cdot x_{i},\bar{t}\cdot y_{j})  \in Al*(i,j) $$

\clearpage
\subsection*{Question 16}
\begin{lstlisting}[
  mathescape,
]
SOL_1(x,y,D):

    (Les insertions dans $\bar{x}$ et $\bar{y}$ se font au debut)
    
    $i \gets |x| $
    $j \gets |y| $

    Tant que i > 0 et j > 0:
        Si $D[i,j] =  D[i-1,j-1] + c_{sub}(x_{i},y_{j})$:
            $\bar{x} \gets x_{i}$
            $\bar{y} \gets y_{j}$
            $i \gets i-1$
            $j \gets j-1$
        Sinon:
            Si $D[i,j] =  D[i,j-1] + c_{ins}$:
                $\bar{x} \gets -$
                $\bar{y} \gets y_{j}$
                $j \gets j-1$
            Sinon:
                $\bar{x} \gets x_{i}$
                $\bar{y} \gets -$
                $i \gets i-1$
    fin tant que
                
    Tant que i > 0 :
        $\bar{x} \gets x_{i}$
        $\bar{y} \gets -$
        $i \gets i-1$
    fin tant que
        
    Tant que j > 0 :
        $\bar{x} \gets -$
        $\bar{y} \gets y_{j}$
        $j \gets j-1$
    fin tant que
    
    return $(\bar{x}, \bar{y})$
            
\end{lstlisting}

\clearpage

\subsection*{Question 17}
SOL\_1 est en $O(n+m)$ et DIST\_1 en $O(max(n,m)^{2})$, ainsi l'éxecution séquentielle de ces deux algorithmes a une complexité temporelle de  $O(max(n,m)^{2})$.

\subsection*{Question 18}
Ces fonctions utilisent des chaines et tableaux de taille allant jusqu'à n+m et un tableau de taille n*m. La complexité spatiale totale est donc $O(nm)$.

\subsection*{Question 19}
Lorsqu'on remplit la ligne i du tableau, on n'a besoin de regarder uniquement 3 cases : D(i-1,j-1), D(i-1,j) et D(i,j-1) (cf l'algorithme).
Ainsi, on n'utilise pas les lignes $D(i',j)_{(i'<i-1, j\in[1..m])}$

\subsection*{Question 20}
\begin{lstlisting}[
  mathescape,
]
DIST_2(x,y):
    Pour j allant de 0 a m:
        $prec[j] \gets j * c_{ins} $
    fin pour
    
    Pour i allant de 1 a n:
    
        Pour j allant de 0 a m:
        
            Si j = 0:
                $D[0] = i * c_{del}$
                
            Sinon:
                c $\gets prec[j-1] + c_{sub}(x_{i},y_{j})$
                    
                Si prec[j]+ $c_{del}$ < c:
                    c $\gets prec[j]+ c_{del}$

                Si D[j-1] + $c_{ins}$ < c:
                    c $\gets D[j-1] + c_{ins}$
                        
                D[j] $\gets$ c
        fin pour
        $prec \gets D$
    fin pour
    
    return D

\end{lstlisting}



\end{document}

